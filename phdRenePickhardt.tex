%%%MAKE TODO COMMAND!

\documentclass[•]{book}
\title{Effective and efficient top-$k$ retrieval.}
\author{Rene Pickhardt (expecting 92 - 123 pages content without structure)}
\date{\today}
\begin{document}
\maketitle
\tableofcontents

%%%%%%%%%%%%%%%%%%%%%%%%%%%%%%%%%%%%%%%%%%%%%%%%%%%%%%%%%%%%%%%%%%%%%%%%%%%%%%%%%%%%%%%%%%%%%%%%%%
%%%
%%%	Introduction
%%% 
%%%%%%%%%%%%%%%%%%%%%%%%%%%%%%%%%%%%%%%%%%%%%%%%%%%%%%%%%%%%%%%%%%%%%%%%%%%%%%%%%%%%%%%%%%%%%%%%%%
\part{Foundations}
\chapter{Introduction (6-10 pages)}
\section{Overview of the topic (3-5)}
Most information retrieval systems \cite{} have a much larger result set for most queries than the user is actually able to look through.
For these cases the notion of relevance of documents in the result set has been introduced. 
This makes it possible to rank the result set according to their given relevance scores.
The following dissertation aims at two things: 

\begin{enumerate}
\item \textbf{Improving the ranking methods for a certain class of queries.}
We introduce generalized language models which have a large variety of applications\cite{own:typology:2013}. 
The application we studied is a next word prediction system which in which we aim to predict the next word a user is about to enter into a computer system like a smart phone. 
Another ranking method we improved is the one of personalized search in graphs\cite{own:graphSearch:2013}.
Here we want to suggest query words to the user matching an entered prefix and taking into account his social context.

\item \textbf{Indexing techniques for several information retrieval systems.}
Let us assume for a moment that we would have the perfect algorithm --- whatever this means --- to obtain relevance scores for the documents in our collection.
We would still have to create fast and scalable indices such that the queries can be answered efficiently and quickly.
We introduce indexing methods for temporal ordered top $k$ joins \cite{own:graphity:2012} which can be found in social networking applications like Facebook or Twitter.
Next we present a trie and interval heap based index for scored prefix queries\cite{own:prefix:2013}. 
This is necessary to support efficient retrieval of language and generalized language models and is used in next word prediction settings like the above mentioned smart phones or in systems like the auto completion service of search queries by Google.
\end{enumerate}

While improving information retrieval algorithms we present precision and recall experiments as well as measuring the mean reciprocal rank where we compared our results with the state of the art base lines. 
We also studied other metrics like normalized key stroke savings that are very specific to the next word prediction problem.

For the data base indices we proved the runtime of CRUD operations of our solutions in a rigour way. 
Additionally we conducted empirical evaluations on various data sets of our indices against the state of the art baselines to demonstrate that the proposed run times can actually be implemented and achieved.

In all cases we demonstrate the superiority of our algorithms which are all available as open source programs together with the used data sets under an open licence.
\section{Running example probably Facebook or even Wikipedia (1-2)}

\section{structure of the text (1-2)}

\section{Acknowledgements (1)}
\begin{itemize}
\item The community of the web providing this large mine of information. Especially the developers and everyone who publishes under an open licence and provides services. In particular the wikipedia foundation and stack exchange. 
\item My students from whom I learnt most
\item Steffen and Institute for creating and providing such a good environment
\item advisor Steffen, Thomas, Gerd.
\item Co authors for great discussion and good spirit (Thomas, Jonas, Till, Paul, Heinrich, Martin,)
\item readers of my blog (reading club, neo4j community, graph devroom)
\item External researchers from various conferences.
\item friends
\item family 

\end{itemize}

%%%%%%%%%%%%%%%%%%%%%%%%%%%%%%%%%%%%%%%%%%%%%%%%%%%%%%%%%%%%%%%%%%%%%%%%%%%%%%%%%%%%%%%%%%%%%%%%%%
%%%
%%%	Definitions and Overview
%%% 
%%%%%%%%%%%%%%%%%%%%%%%%%%%%%%%%%%%%%%%%%%%%%%%%%%%%%%%%%%%%%%%%%%%%%%%%%%%%%%%%%%%%%%%%%%%%%%%%%%
\chapter{Foundations of the thesis (20 pages)}
\section{Effective methods in Information retrieval}
\subsection{Basic Definitions in Information Retrieval}
Thomas script on information retrieval could be of relevance here.
\begin{itemize}
\item query / keyword query / semantic query
\item document
\item corpus
\item ranking
\item relevance
\item context
\end{itemize}
\subsection{Information need (english term for informationsbedurfnis?)}
\subsection{Relvance Problem}
\subsection{Standard methods Precision / Recall / architectures.}
\section{efficient and high scalable indices for information retrieval}
\subsection{requirements for an information retrieval index}
\begin{itemize}
\item fastest possible retrieval
\item maintainable (fast updates)
\item low redundancy
\end{itemize}
\subsection{common problems with data bases}
\subsection{useful algorithms and data structures}
\begin{itemize}
\item Many stuff is from Mehlhorn \cite{rw:algodat:Mehlhorn:1990} maybe also LEDA
\item priority queues (on Fibonacci heaps)
\item min max heaps
\item multidimensional trees
\item tries
\end{itemize}


%%%%%%%%%%%%%%%%%%%%%%%%%%%%%%%%%%%%%%%%%%%%%%%%%%%%%%%%%%%%%%%%%%%%%%%%%%%%%%%%%%%%%%%%%%%%%%%%%%
%%%
%%%	Top k queries in information retrieval is a two sided problem
%%% 
%%%%%%%%%%%%%%%%%%%%%%%%%%%%%%%%%%%%%%%%%%%%%%%%%%%%%%%%%%%%%%%%%%%%%%%%%%%%%%%%%%%%%%%%%%%%%%%%%%
\chapter{Top $k$ queries in information retrieval is a two sided problem (3 pages)}
is this the correct spot or does this have to be after related work because we can work this out from the related work?
\section{Effectiveness: Calculating the ranking for the top $k$ set}
\section{Efficiency: Building indices for fast retrieval of the top $k$ set}

%%%%%%%%%%%%%%%%%%%%%%%%%%%%%%%%%%%%%%%%%%%%%%%%%%%%%%%%%%%%%%%%%%%%%%%%%%%%%%%%%%%%%%%%%%%%%%%%%%
%%%
%%%	Problems studied in this dissertation increasing effectiveness
%%% 
%%%%%%%%%%%%%%%%%%%%%%%%%%%%%%%%%%%%%%%%%%%%%%%%%%%%%%%%%%%%%%%%%%%%%%%%%%%%%%%%%%%%%%%%%%%%%%%%%%
\part{Calculating the ranking for top $k$ queries}

%%%%%%%%%%%%%%%%%%%%%%%%%%%%%%%%%%%%%%%%%%%%%%%%%%%%%%%%%%%%%%%%%%%%%%%%%%%%%%%%%%%%%%%%%%%%%%%%%%
%%%
%%%	generalized language models
%%% 
%%%%%%%%%%%%%%%%%%%%%%%%%%%%%%%%%%%%%%%%%%%%%%%%%%%%%%%%%%%%%%%%%%%%%%%%%%%%%%%%%%%%%%%%%%%%%%%%%%
\chapter{generalized language models (10 - 15 pages)}
\begin{itemize}
\item paper almost done. 
\item might need some resubmissions but then lets go.
\item a current bachlor thesis goes into this direction
\end{itemize}

%%%%%%%%%%%%%%%%%%%%%%%%%%%%%%%%%%%%%%%%%%%%%%%%%%%%%%%%%%%%%%%%%%%%%%%%%%%%%%%%%%%%%%%%%%%%%%%%%%
%%%
%%%	personalized search / autocomplete rankings
%%% 
%%%%%%%%%%%%%%%%%%%%%%%%%%%%%%%%%%%%%%%%%%%%%%%%%%%%%%%%%%%%%%%%%%%%%%%%%%%%%%%%%%%%%%%%%%%%%%%%%%
\chapter{Personalized search / autocomplete rankings (10 - 15 pages)}
\begin{itemize}
\item a current bachlor thesis goes into this direction
\end {itemize}


%%%%%%%%%%%%%%%%%%%%%%%%%%%%%%%%%%%%%%%%%%%%%%%%%%%%%%%%%%%%%%%%%%%%%%%%%%%%%%%%%%%%%%%%%%%%%%%%%%
%%%
%%%	Problems studied in this dissertation increasing efficiency
%%% 
%%%%%%%%%%%%%%%%%%%%%%%%%%%%%%%%%%%%%%%%%%%%%%%%%%%%%%%%%%%%%%%%%%%%%%%%%%%%%%%%%%%%%%%%%%%%%%%%%%
\part{Building efficient indices for top $k$ queries}
%%%%%%%%%%%%%%%%%%%%%%%%%%%%%%%%%%%%%%%%%%%%%%%%%%%%%%%%%%%%%%%%%%%%%%%%%%%%%%%%%%%%%%%%%%%%%%%%%%
%%%
%%%	general difficulty
%%% 
%%%%%%%%%%%%%%%%%%%%%%%%%%%%%%%%%%%%%%%%%%%%%%%%%%%%%%%%%%%%%%%%%%%%%%%%%%%%%%%%%%%%%%%%%%%%%%%%%%
\chapter{arbitrary top $k$ queries are inefficient (3 - 5 pages)}
e.g. top $k$ aggregation joins might need a full materialization of the join and a complete walk through the result set to calculate the top $k$ join set. 

Could be moved to foundations?
%%%%%%%%%%%%%%%%%%%%%%%%%%%%%%%%%%%%%%%%%%%%%%%%%%%%%%%%%%%%%%%%%%%%%%%%%%%%%%%%%%%%%%%%%%%%%%%%%%
%%%
%%%	scored top k prefix queries
%%% 
%%%%%%%%%%%%%%%%%%%%%%%%%%%%%%%%%%%%%%%%%%%%%%%%%%%%%%%%%%%%%%%%%%%%%%%%%%%%%%%%%%%%%%%%%%%%%%%%%%
\chapter{Indices for top $k$ scored prefix queries (10 - 15 pages)}
\begin{itemize}
\item Powering generalized language models or any form of auto completion. 
\item Problem solved. 
\item nothing implemented.
\item no paper written.
\end{itemize}

%%%%%%%%%%%%%%%%%%%%%%%%%%%%%%%%%%%%%%%%%%%%%%%%%%%%%%%%%%%%%%%%%%%%%%%%%%%%%%%%%%%%%%%%%%%%%%%%%%
%%%
%%%	graphity indices for top $k$ joins on temporal ordered data
%%% 
%%%%%%%%%%%%%%%%%%%%%%%%%%%%%%%%%%%%%%%%%%%%%%%%%%%%%%%%%%%%%%%%%%%%%%%%%%%%%%%%%%%%%%%%%%%%%%%%%%
\chapter{Indices for top $k$ joins on temporal ordered data (10 - 15 pages)}
\begin{itemize}
\item Graphity and the principles behind this. 
\item paper done 
\item better presentation possible. 
\end{itemize}
%%%%%%%%%%%%%%%%%%%%%%%%%%%%%%%%%%%%%%%%%%%%%%%%%%%%%%%%%%%%%%%%%%%%%%%%%%%%%%%%%%%%%%%%%%%%%%%%%%
%%%
%%%	solving holger bast problem faster
%%% 
%%%%%%%%%%%%%%%%%%%%%%%%%%%%%%%%%%%%%%%%%%%%%%%%%%%%%%%%%%%%%%%%%%%%%%%%%%%%%%%%%%%%%%%%%%%%%%%%%%
\chapter{solving Holger Bast problem more elegant (10 - 15 pages)}
\begin{itemize}
\item To be done (no ideas yet! but it can't be difficult).
\item Holger Bast solves the problem of instant document retrieval while keyword typing \cite{RW:Bast:2006} (also taking into account all possible auto completions which are way to many and make the result set irrelevant but which is needed for his kind of algorithm). 
\item we could also include and propose the personalized ranking into the problem set.
\end{itemize}

%%%%%%%%%%%%%%%%%%%%%%%%%%%%%%%%%%%%%%%%%%%%%%%%%%%%%%%%%%%%%%%%%%%%%%%%%%%%%%%%%%%%%%%%%%%%%%%%%%
%%%
%%%	Summary, lessons learnt and future work
%%% 
%%%%%%%%%%%%%%%%%%%%%%%%%%%%%%%%%%%%%%%%%%%%%%%%%%%%%%%%%%%%%%%%%%%%%%%%%%%%%%%%%%%%%%%%%%%%%%%%%%
\part{Summary, lessons learnt and future work}

%%%%%%%%%%%%%%%%%%%%%%%%%%%%%%%%%%%%%%%%%%%%%%%%%%%%%%%%%%%%%%%%%%%%%%%%%%%%%%%%%%%%%%%%%%%%%%%%%%
%%%
%%%	Conclusions / contributions
%%% 
%%%%%%%%%%%%%%%%%%%%%%%%%%%%%%%%%%%%%%%%%%%%%%%%%%%%%%%%%%%%%%%%%%%%%%%%%%%%%%%%%%%%%%%%%%%%%%%%%%
\chapter{Conclusion / Contributions (5 pages)}
\begin{itemize}
\item For well known top $k$ queries it is possible to find efficient indices. This requires large effort. 
\item so far best known solution of some well known problems with rich applications. 
\end{itemize}
%%%%%%%%%%%%%%%%%%%%%%%%%%%%%%%%%%%%%%%%%%%%%%%%%%%%%%%%%%%%%%%%%%%%%%%%%%%%%%%%%%%%%%%%%%%%%%%%%%
%%%
%%%	Future work
%%% 
%%%%%%%%%%%%%%%%%%%%%%%%%%%%%%%%%%%%%%%%%%%%%%%%%%%%%%%%%%%%%%%%%%%%%%%%%%%%%%%%%%%%%%%%%%%%%%%%%%
\chapter{Future work (5 pages)}
\section{developing a complete theory of which kind of queries lead to efficient indices}
\section{proofing that current indices achieve the best possible runtime}

\bibliographystyle{plain}
\bibliography{phdRenePickhardt}

\end{document}
