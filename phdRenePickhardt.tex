%%%MAKE TODO COMMAND!

\documentclass[•]{article}
\title{Ranking techniques in Information Retrieval and Data base indices for top $k$ queries}
\author{Rene Pickhardt}
\date{\today}
\begin{document}
\maketitle
\tableofcontents
\pagebreak
\section{Prologue}
Most information retrieval systems \cite{} have a much larger result set for most queries than the user is actually able to look through.
For these cases the notion of relevance of documents in the result set has been introduced. 
This makes it possible to rank the result set according to their given relevance scores.
The following dissertation aims at two things: 

\begin{enumerate}
\item \textbf{Improving the ranking methods for a certain class of queries.}
We introduce generalized language models which have a large variety of applications\cite{own:typology:2013}. 
The application we studied is a next word prediction system which in which we aim to predict the next word a user is about to enter into a computer system like a smart phone. 
Another ranking method we improved is the one of personalized search in graphs\cite{own:graphSearch:2013}.
Here we want to suggest query words to the user matching an entered prefix and taking into account his social context.

\item \textbf{Indexing techniques for several information retrieval systems.}
Let us assume for a moment that we would have the perfect algorithm --- whatever this means --- to obtain relevance scores for the documents in our collection.
We would still have to create fast and scalable indices such that the queries can be answered efficiently and quickly.
We introduce indexing methods for temporal ordered top $k$ joins \cite{own:graphity:2012} which can be found in social networking applications like Facebook or Twitter.
Next we present a trie and interval heap based index for scored prefix queries\cite{own:prefix:2013}. 
This is necessary to support efficient retrieval of language and generalized language models and is used in next word prediction settings like the above mentioned smart phones or in systems like the auto completion service of search queries by Google.
\end{enumerate}

While improving information retrieval algorithms we present precision and recall experiments as well as measuring the mean reciprocal rank where we compared our results with the state of the art base lines. 
We also studied other metrics like normalized key stroke savings that are very specific to the next word prediction problem.

For the data base indices we proved the runtime of CRUD operations of our solutions in a rigour way. 
Additionally we conducted empirical evaluations on various data sets of our indices against the state of the art baselines to demonstrate that the proposed run times can actually be implemented and achieved.

In all cases we demonstrate the superiority of our algorithms which are all available as open source programs together with the used data sets under an open licence.
%%%%%%%%%%%%%%%%%%%%%%%%%%%%%%%%%%%%%%%%%%%%%%%%%%%%%%%%%%%%%%%%%%%%%%%%%%%%%%%%%%%%%%%%%%%%%%%%%%
%%%
%%%	Introduction
%%% 
%%%%%%%%%%%%%%%%%%%%%%%%%%%%%%%%%%%%%%%%%%%%%%%%%%%%%%%%%%%%%%%%%%%%%%%%%%%%%%%%%%%%%%%%%%%%%%%%%%
\pagebreak
\section{Introduction}
probably missing some subsections here.
\subsection{Search engine as a running example}
\subsubsection{auto completion}
\begin{itemize}
\item fast indices are required
\item relevant terms have to be suggested (ranking not so important simple techniques will work)
\end{itemize}

\subsubsection{Visualization of search results}
\begin{itemize}
\item relevant documents have to be displayed (ranking very complex)
\item fast indices (search engine should be fast)
\end{itemize}

\subsubsection{Personalization}
\begin{itemize}
\item relevance comes from the ego network
\item indices have to be personalized which might be more complex?
\end{itemize}

\subsubsection{Real time search of Twitter / news}
\begin{itemize}
\item only a fast indices (search engine should be fast)
\item ranking comes from the temporal structure
\item filtering comes from the ego network
\item filtering (situation becomes different in realtime keyword search of tweets)
\end{itemize}

\subsubsection{Auction system in advertising}
just for a complete overview on the search process
\begin{itemize}
\item relevance of ads is easy
\item speed is important since delivery of ads can't take more time than the rest
\end{itemize}
\subsection{Outline of the dissertation}


%%%%%%%%%%%%%%%%%%%%%%%%%%%%%%%%%%%%%%%%%%%%%%%%%%%%%%%%%%%%%%%%%%%%%%%%%%%%%%%%%%%%%%%%%%%%%%%%%%
%%%
%%%	Definitions and Overview
%%% 
%%%%%%%%%%%%%%%%%%%%%%%%%%%%%%%%%%%%%%%%%%%%%%%%%%%%%%%%%%%%%%%%%%%%%%%%%%%%%%%%%%%%%%%%%%%%%%%%%%
\pagebreak
\section{Definitions / Overview}
\subsection{Information retrieval}
\subsubsection{Basic Definitions in Information Retrieval}
Thomas script on information retrieval could be of relevance here.
\begin{itemize}
\item query / keyword query / semantic query
\item document
\item corpus
\item ranking
\item relevance
\item context
\end{itemize}
\subsubsection{Information need (english term for informationsbedurfnis?)}
\subsubsection{Relvance Problem}
\subsubsection{Standard methods Precision / Recall / architectures.}
\subsection{data base technology}
\subsubsection{requirements for a data base index}
\begin{itemize}
\item fastest possible retrieval
\item maintainable
\item low redundancy
\end{itemize}
\subsubsection{common problems with data bases}
\subsubsection{fundamental algorithms and data structures}
\begin{itemize}
\item priority queues (on Fibonacci heaps)
\item multidimensional trees
\item tries
\end{itemize}


%%%%%%%%%%%%%%%%%%%%%%%%%%%%%%%%%%%%%%%%%%%%%%%%%%%%%%%%%%%%%%%%%%%%%%%%%%%%%%%%%%%%%%%%%%%%%%%%%%
%%%
%%%	Top k queries in information retrieval is a two sided problem
%%% 
%%%%%%%%%%%%%%%%%%%%%%%%%%%%%%%%%%%%%%%%%%%%%%%%%%%%%%%%%%%%%%%%%%%%%%%%%%%%%%%%%%%%%%%%%%%%%%%%%%
\pagebreak
\section{Top $k$ queries in information retrieval is a two sided problem}
is this the correct spot or does this have to be after related work because we can work this out from the related work?
\subsection{Calculating the ranking for the top $k$ set}
\subsection{Building indices for fast retrieval of the top $k$ set}

%%%%%%%%%%%%%%%%%%%%%%%%%%%%%%%%%%%%%%%%%%%%%%%%%%%%%%%%%%%%%%%%%%%%%%%%%%%%%%%%%%%%%%%%%%%%%%%%%%
%%%
%%%	Related work (creating a reading club)
%%% 
%%%%%%%%%%%%%%%%%%%%%%%%%%%%%%%%%%%%%%%%%%%%%%%%%%%%%%%%%%%%%%%%%%%%%%%%%%%%%%%%%%%%%%%%%%%%%%%%%%
\section{Related Work}

%%%%%%%%%%%%%%%%%%%%%%%%%%%%%%%%%%%%%%%%%%%%%%%%%%%%%%%%%%%%%%%%%%%%%%%%%%%%%%%%%%%%%%%%%%%%%%%%%%
%%%
%%%	Problems studied in this dissertation
%%% 
%%%%%%%%%%%%%%%%%%%%%%%%%%%%%%%%%%%%%%%%%%%%%%%%%%%%%%%%%%%%%%%%%%%%%%%%%%%%%%%%%%%%%%%%%%%%%%%%%%
\pagebreak
\section{Problems studied in this dissertation}

\subsection{Building efficient indices for top $k$ queries}
\subsubsection{arbitrary top $k$ queries are inefficient}
e.g. top $k$ aggregation joins might need a full materialization of the join and a complete walk through the result set to calculate the top $k$ join set.
\subsubsection{Indices for top $k$ scored prefix queries}
\begin{itemize}
\item Powering generalized language models or any form of auto completion. 
\item Problem solved. 
\item nothing implemented.
\item no paper written.
\end{itemize}


\subsubsection{Indices for top $k$ joins on temporal ordered data}
\begin{itemize}
\item Graphity and the principles behind this. 
\item paper done 
\item better presentation possible. 
\end{itemize}

\subsubsection{solving Holger Bast problem more elegant}
\begin{itemize}
\item To be done (no ideas yet! but it can't be difficult).
\item Holger Bast solves the problem of instant document retrieval while keyword typing \cite{RW:Bast:2006} (also taking into account all possible auto completions which are way to many and make the result set irrelevant but which is needed for his kind of algorithm). 
\item we could also include and propose the personalized ranking into the problem set.
\end{itemize}

\subsection{Calculating the ranking for top $k$ queries}

\subsubsection{generalized language models}
\begin{itemize}
\item paper almost done. 
\item might need some resubmissions but then lets go.
\end{itemize}

\subsubsection{The really generalized language models}
\begin{itemize}
\item a current bachlor thesis goes into this direction
\end {itemize}

\subsubsection{Personalized search / autocomplete rankings}
\begin{itemize}
\item a current bachlor thesis goes into this direction
\end {itemize}

%%%%%%%%%%%%%%%%%%%%%%%%%%%%%%%%%%%%%%%%%%%%%%%%%%%%%%%%%%%%%%%%%%%%%%%%%%%%%%%%%%%%%%%%%%%%%%%%%%
%%%
%%%	Conclusions / contributions
%%% 
%%%%%%%%%%%%%%%%%%%%%%%%%%%%%%%%%%%%%%%%%%%%%%%%%%%%%%%%%%%%%%%%%%%%%%%%%%%%%%%%%%%%%%%%%%%%%%%%%%
\pagebreak
\section{Conclusion / Contributions}
\begin{itemize}
\item For well known top $k$ queries it is possible to find efficient indices. This requires large effort. 
\item so far best known solution of some well known problems with rich applications. 
\end{itemize}
%%%%%%%%%%%%%%%%%%%%%%%%%%%%%%%%%%%%%%%%%%%%%%%%%%%%%%%%%%%%%%%%%%%%%%%%%%%%%%%%%%%%%%%%%%%%%%%%%%
%%%
%%%	Future work
%%% 
%%%%%%%%%%%%%%%%%%%%%%%%%%%%%%%%%%%%%%%%%%%%%%%%%%%%%%%%%%%%%%%%%%%%%%%%%%%%%%%%%%%%%%%%%%%%%%%%%%
\section{Future work}
\subsubsection{developing a complete theory of which kind of queries lead to efficient indices}
\subsubsection{proofing that current indices achieve the best possible runtime}

\bibliographystyle{plain}
\bibliography{phdRenePickhardt}

\end{document}